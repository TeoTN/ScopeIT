% \pagebreak[4]
% \hspace*{1cm}
% \pagebreak[4]
% \hspace*{1cm}
% \pagebreak[4]

\chapter{Gale-Shapley stable matching algorithm}
\ifpdf
    \graphicspath{{Chapter1/Chapter1Figs/PNG/}{Chapter1/Chapter1Figs/PDF/}{Chapter1/Chapter1Figs/}}
\else
    \graphicspath{{Chapter1/Chapter1Figs/EPS/}{Chapter1/Chapter1Figs/}}
\fi
\marginnote{Terms: perfect matching, instability, etc. Restrictions of G-S algorithm and extensions. Time complexity. Reference to proof of correctness.}
The Stable Matching problem is occurs in a number of disciplines, including mathematics, economics and computer science. The earliest known matching scheme originated in 1952, when National Resident Matching Program was launched in United States, in order to provide reliable and fair method of matching graduating medical students to their first hospital assignments. \cite{nrmp} Independently, the such an algorithm was devised by David Gale and Lloyd Shapley in 1962 \cite{DBLP:books/daglib/0015106}. The most basic variant of the problem it solves is formulated in the following paragraph.

Let ${\mathit{M, W}}$ denote two sets of distinct agents, each of them having some ordered preference list of elements from the opposite set. An outcome of the algorithm is a set of ordered pairs ${\mathit{S = \{(m, w) \in M \times W\}}}$ having a following property:
For pair ${\mathit{(m, w) \in S}}$ there are no elements ${\mathit{m' \in M, w' \in W}}$ such that \textit{m would prefer w' to w} and \textit{w would prefer m' to m}.

\section{Properties of stable matching}

We will be concerned about some variation on the algorithm which can handle following assumptions:
\begin{enumerate}
	\item Number of members within two groups can be different
	\item An element from one group can be matched to multiple elements from the latter one, i.e. for agent \textit{e} there is some arbitrary \textit{n} such that ${\mathit{(e, x_1), ... (e, x_n)  \in S}}$
	\item It is possible that some agents remain without a pair: in such case an applicant would rather remain unemployed than choose any offer, as well as an employer may decide not to employ anyone.
	\item Ties on a preference list are allowed
	\item There are some forbidden pairs
\end{enumerate}


\section{Algorithm design}
Algorithm design is depicted in two subsections:

\subsection{Pseudocode}

\subsection{Implementation details}
% ------------------------------------------------------------------------


%%% Local Variables: 
%%% mode: latex
%%% TeX-master: "../thesis"
%%% End: 
