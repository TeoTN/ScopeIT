%%% Thesis Introduction --------------------------------------------------
\chapter{Introduction}
\ifpdf
    \graphicspath{{Introduction/IntroductionFigs/PNG/}{Introduction/IntroductionFigs/PDF/}{Introduction/IntroductionFigs/}}
\else
    \graphicspath{{Introduction/IntroductionFigs/EPS/}{Introduction/IntroductionFigs/}}
\fi

Information Technology labour market is quite distinct from others in terms of how people tend to seek for a job. Position requirements for a programmer are often very concise and precise, therefore it is feasible to quantify the skills and requirements an applicant has.

In recent years, a handful of job searching websites dedicated solely to programmers emerged on the internet\footnote{Examples of this approach are: www.filltr.pl and www.nofluffjobs.com} and presented a rather unique approach to filtering jobs. It is now preferable to put an emphasis on programming languages, technologies, libraries and frameworks which are involved in the post, rather than let users search by keywords and skim through a block of text within the offer.

\section{Overview}
In this particular case, especially at the beginning of recruitment process, a crucial part of matching an applicant and an employer is to ensure that some basic requirements are fulfilled. This gives way to providing a job search automation.

The both sides of the market can establish some matching based on applicant's and employer's requirements and offerings, however it cannot be guaranteed that participants do not change their mind. Informally, Gale-Shapley \textit{stable matching algorithm} is an algorithm which outputs the most preferable pairs from two distinct sets, so that no other pair would rather be formed. The algorithm and its application is thoroughly discussed in the first chapter.

The whole process of finding a suitable job ought to be straightforward and therefore preferably the user to machine interaction should be kept at the bare minimum. Possible use cases of the web application created over the course of the thesis are widely described in the second chapter.

Not a single website might have been built without some technologies. An overview of programming languages, frameworks and libraries used in the project will be presented in the third chapter. Its general structure is also explained in this part, as well as the motivation behind it.

In the last chapter, guidelines to the installation process are presented in line with a few possible further improvements which can be introduced.

\section{Motivation}
The need for automation of job searching process is motivated by a variety of problems it may solve. The first advantage of using the stable matching algorithm is that it forms pairs of employers and applicants such that there is neither employer nor applicant which would rather work with someone else and the one also prefers that matching.

One can easily imagine the situation in which an applicant is offered a job by \textit{company A}, which at first is accepted, however in the near future \textit{company B} may also offer a job to the applicant. As a result the candidate may decide to turn down the first proposal and incline towards the second one, which may increase the company A costs of recruitment process. Also the opposite situation is possible when it is the company which finds more suitable candidate for the post.

Furthermore, finding an IT job may turn out to be a daunting task, due to the fact that one has to skim through a vast number of announcements, yet rarely is the list represented and filtered only by a quantifiable set of skills and requirements. The approach presented in the following project will generate propositions for both companies and applicants without an intense involvement. In turn, this will reduce the process of finding a suitable job to a mere procedure of filling in the profile.

\section{Related work}
LinkedIn, Filttr, Nofluffjobs
Algortihms?

\section{}
%%% ----------------------------------------------------------------------


%%% Local Variables: 
%%% mode: latex
%%% TeX-master: "../thesis"
%%% End: 
